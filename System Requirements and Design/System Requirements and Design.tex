\documentclass[a4paper,10pt]{article}

\usepackage[margin=2cm]{geometry}
\usepackage{graphicx}
\usepackage{amsmath}
\usepackage{array}
\usepackage{hyperref}
\usepackage[all]{hypcap}
\usepackage{listings}
\lstdefinestyle{TerminalStyle}{
  language=bash,
  basicstyle=\small\sffamily,
  numbers=left,
  numberstyle=\tiny,
  numbersep=3pt,
  frame=tb,
  columns=fullflexible,
  linewidth=0.9\linewidth,
  xleftmargin=0.1\linewidth
}
\lstdefinestyle{HtmlStyle}{
  language=html,
  basicstyle=\small\sffamily,
  numbers=left,
  numberstyle=\tiny,
  numbersep=3pt,
  frame=tb,
  columns=fullflexible,
  linewidth=0.9\linewidth,
  xleftmargin=0.1\linewidth
}
\lstdefinestyle{OutputStyle}{
  language=html,
  basicstyle=\small\sffamily,
  frame=tb,
  columns=fullflexible,
  linewidth=0.9\linewidth,
  xleftmargin=0.1\linewidth
}

\setlength{\parindent}{0pt}
\setlength{\parskip}{1ex plus 0.5ex minus 0.2ex}
\title{\includegraphics[width=12cm]{Eeufeeslogo.jpg} \\
       Department of Computer Science \\
       University of Pretoria \\
       \vspace{0.5cm}
       Software Engineering\\
       COS301 Main Project \\
       \vspace{0.5cm}
       \begin{large} \textbf{Team CodeX}\\ ReRoute Systems\end{large}}

\date{} 
\author{	Bondjobo, Jocelyn 		13232852 		\\
		Malangu, Daniel		13315120		\\
		Kirker, Tim			11152402		\\
		Hammond, Eunice		13222563		\\
		Burgers, Heinrich		15059538		\\
}

\begin{document}
\maketitle
\thispagestyle{empty}
\clearpage

\newpage
\pagenumbering{roman}
\thispagestyle{empty}
\tableofcontents
\clearpage

\newpage
\pagenumbering{arabic}

\section{Introduction}

	\subsection{Background} 
	Reroute Systems is a software company with different in-house developed applications. The Purchase Management System application is the main application and mainly active in the 		pharmaceutical space. The main functionality of the system is routing of request for products from account holders to the various wholesalers / suppliers, receiving the result of the order and route answer back to account holder.
	\subsection{Purpose} 	
	For the account holder to request a product he/she must search for it against the database with all the product information, the challenge is that each wholesalers / suppliers can name/describe the product differently and when the account holder do the search against the master product list the same product must be displayed across all wholesalers / suppliers using the link between master product list and the different wholesalers / suppliers product list (the user compare the prices of same product across the wholesalers / suppliers before making a decision)
	\subsection{Scope} 
	\subsection{Definitions, Acronyms, and Abbreviations} 
	\subsection{References} 
	\subsection{Overview} 

\section{Overall Description}

	\subsection{Product Perspective}
	
		\subsubsection{System Interfaces}

		\subsubsection{User Interfaces}

		\subsubsection{Hardware Interfaces}

		\subsubsection{Software Interfaces}

		\subsubsection{Communications Interfaces}
	
		\subsubsection{Memory}

		\subsubsection{Operations}

		\subsubsection{Site Adaptation Requirements}
		
	\subsection{Product Functions}			
					
	\subsection{User Characteristics}

	\subsection{Constraints}

	\subsection{Assumptions and Dependencies}

		
	\section{Specific Acquirements}
This section gives a detailed description of the system requirements. It describes all the functional as well as the quality requirements of the system.

	\subsection{External Interface Requirements}

                 \subsubsection{User Interfaces}

                 \subsubsection{Software Interfaces}
The application will run on the Android operating system, specifically version 4.0. and upwards.It will also run on iOS operating system version 7.3 and above.

	\subsection{Requirements}
	\subsubsection{Functional Requirements} 
	1.	Fuzzy Search:\\\\
	This program must be able to locate and return products that are related to the searched term. It should be able to return the 	same product if various ways of writing the name is used.\\\\
	
	2.	Fast Run-time:\\\\
	Because there are very large amount of data to be searched, it is vital for our program to be very efficient.  Even though this could be a non-functional requirement, it is vital for this program to be as fast and efficient as possible.\\\\
	
	3.	Handle Concurrent Users:\\\\
	This program will be used by many people, in many different fields simultaneously. It is therefore very important for it to be able to handle multiple users at the same time.\\\\
	
	4.	Machine Learning:\\\\
	In order for this program to run efficiently it must be able to adapt to the needs and habits of its users. It should be able to learn search patterns and learn from previous searches.\\\\
	
	\subsubsection{Non-Functional Requirements}\\\\
	1.	Predictive Typing:\\\\
	Predictive typing is when the program suggests a possible solution as the user is typing. This is often based on previous searches and most common searches. This will help improve the user experience. \\\\

	2.	Return Generic Medicine:\\\\
	In some cases it will be handy if the program suggested the different generic medicine for the medicine you searched for. This could help users find the product they need even if they are unsure of the name.
	
	3.	Robustness:\\\\
	This product should be able to adapt to different users in various fields as well as to different habits, interfaces, environments and needs of users.  
	
	4.	Storing and using search history: 
	Being able to store the search history will increase the user experience, help with predictive typing and improve the programs ability to learn and adapt.

	
	\subsubsection{Use case prioritization} 
		\begin{enumerate} 
		\item \textbf{Critical} 
			\begin{itemize} 
				\item Login
				\item Retrieve Whole Seller Accounts 
				\item Search Product 
				
				\textbf{Pre-condition: } The user has to have been logged in successfully.  \\
				\textbf{Post-condition: } The user must be logged out. \\
				\begin{center}
				\begin{tabular}{ |p{8cm}|p{8cm}| }
				 \hline
  				\textbf{Actor:} User & \textbf{System:} Logout \\
				 \hline
				 0. User clicks on the profile button & - \\
				 \hline
				- & 2. System retriveies profile information and displays it\\
				 \hline
				 3. User clicks On the logout button on the profile tab & - \\
				 \hline
				- & 4. Signal is sent to the server and the system logs the user out\\
				 \hline
				\end{tabular}
				\end{center}
				\item Logout
			\end{itemize} 
		\item \textbf{Important} 

		\item \textbf{Nice to have} 
		\end{enumerate} 

	\subsection{Performance Requirements}

	\subsection{Design Constraints}

	\subsection{Software System Attributes}

	\subsection{Other Requirements}

\subsubsection{Quality requirements}

\clearpage

\section{Open Issues}
\subsection {GitHub Repository}
\includegraphics[width=12cm]{CodeX_logo.jpg} \\
Team CodeX Repository: \url{https://github.com/josephbondjobo/CodeX}

This repository contains:
\begin{itemize}
\item All work done by team members.
\end{itemize}



\newpage
\clearpage
\addcontentsline{toc}{section}{References}

\end{document}
